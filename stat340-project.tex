% Options for packages loaded elsewhere
\PassOptionsToPackage{unicode}{hyperref}
\PassOptionsToPackage{hyphens}{url}
%
\documentclass[
]{article}
\title{STAT340 Final Report: Gender Base Salary Gap in STEM Field}
\author{}
\date{\vspace{-2.5em}12/02/2021}

\usepackage{amsmath,amssymb}
\usepackage{lmodern}
\usepackage{iftex}
\ifPDFTeX
  \usepackage[T1]{fontenc}
  \usepackage[utf8]{inputenc}
  \usepackage{textcomp} % provide euro and other symbols
\else % if luatex or xetex
  \usepackage{unicode-math}
  \defaultfontfeatures{Scale=MatchLowercase}
  \defaultfontfeatures[\rmfamily]{Ligatures=TeX,Scale=1}
\fi
% Use upquote if available, for straight quotes in verbatim environments
\IfFileExists{upquote.sty}{\usepackage{upquote}}{}
\IfFileExists{microtype.sty}{% use microtype if available
  \usepackage[]{microtype}
  \UseMicrotypeSet[protrusion]{basicmath} % disable protrusion for tt fonts
}{}
\makeatletter
\@ifundefined{KOMAClassName}{% if non-KOMA class
  \IfFileExists{parskip.sty}{%
    \usepackage{parskip}
  }{% else
    \setlength{\parindent}{0pt}
    \setlength{\parskip}{6pt plus 2pt minus 1pt}}
}{% if KOMA class
  \KOMAoptions{parskip=half}}
\makeatother
\usepackage{xcolor}
\IfFileExists{xurl.sty}{\usepackage{xurl}}{} % add URL line breaks if available
\IfFileExists{bookmark.sty}{\usepackage{bookmark}}{\usepackage{hyperref}}
\hypersetup{
  pdftitle={STAT340 Final Report: Gender Base Salary Gap in STEM Field},
  hidelinks,
  pdfcreator={LaTeX via pandoc}}
\urlstyle{same} % disable monospaced font for URLs
\usepackage[margin=1in]{geometry}
\usepackage{color}
\usepackage{fancyvrb}
\newcommand{\VerbBar}{|}
\newcommand{\VERB}{\Verb[commandchars=\\\{\}]}
\DefineVerbatimEnvironment{Highlighting}{Verbatim}{commandchars=\\\{\}}
% Add ',fontsize=\small' for more characters per line
\usepackage{framed}
\definecolor{shadecolor}{RGB}{248,248,248}
\newenvironment{Shaded}{\begin{snugshade}}{\end{snugshade}}
\newcommand{\AlertTok}[1]{\textcolor[rgb]{0.94,0.16,0.16}{#1}}
\newcommand{\AnnotationTok}[1]{\textcolor[rgb]{0.56,0.35,0.01}{\textbf{\textit{#1}}}}
\newcommand{\AttributeTok}[1]{\textcolor[rgb]{0.77,0.63,0.00}{#1}}
\newcommand{\BaseNTok}[1]{\textcolor[rgb]{0.00,0.00,0.81}{#1}}
\newcommand{\BuiltInTok}[1]{#1}
\newcommand{\CharTok}[1]{\textcolor[rgb]{0.31,0.60,0.02}{#1}}
\newcommand{\CommentTok}[1]{\textcolor[rgb]{0.56,0.35,0.01}{\textit{#1}}}
\newcommand{\CommentVarTok}[1]{\textcolor[rgb]{0.56,0.35,0.01}{\textbf{\textit{#1}}}}
\newcommand{\ConstantTok}[1]{\textcolor[rgb]{0.00,0.00,0.00}{#1}}
\newcommand{\ControlFlowTok}[1]{\textcolor[rgb]{0.13,0.29,0.53}{\textbf{#1}}}
\newcommand{\DataTypeTok}[1]{\textcolor[rgb]{0.13,0.29,0.53}{#1}}
\newcommand{\DecValTok}[1]{\textcolor[rgb]{0.00,0.00,0.81}{#1}}
\newcommand{\DocumentationTok}[1]{\textcolor[rgb]{0.56,0.35,0.01}{\textbf{\textit{#1}}}}
\newcommand{\ErrorTok}[1]{\textcolor[rgb]{0.64,0.00,0.00}{\textbf{#1}}}
\newcommand{\ExtensionTok}[1]{#1}
\newcommand{\FloatTok}[1]{\textcolor[rgb]{0.00,0.00,0.81}{#1}}
\newcommand{\FunctionTok}[1]{\textcolor[rgb]{0.00,0.00,0.00}{#1}}
\newcommand{\ImportTok}[1]{#1}
\newcommand{\InformationTok}[1]{\textcolor[rgb]{0.56,0.35,0.01}{\textbf{\textit{#1}}}}
\newcommand{\KeywordTok}[1]{\textcolor[rgb]{0.13,0.29,0.53}{\textbf{#1}}}
\newcommand{\NormalTok}[1]{#1}
\newcommand{\OperatorTok}[1]{\textcolor[rgb]{0.81,0.36,0.00}{\textbf{#1}}}
\newcommand{\OtherTok}[1]{\textcolor[rgb]{0.56,0.35,0.01}{#1}}
\newcommand{\PreprocessorTok}[1]{\textcolor[rgb]{0.56,0.35,0.01}{\textit{#1}}}
\newcommand{\RegionMarkerTok}[1]{#1}
\newcommand{\SpecialCharTok}[1]{\textcolor[rgb]{0.00,0.00,0.00}{#1}}
\newcommand{\SpecialStringTok}[1]{\textcolor[rgb]{0.31,0.60,0.02}{#1}}
\newcommand{\StringTok}[1]{\textcolor[rgb]{0.31,0.60,0.02}{#1}}
\newcommand{\VariableTok}[1]{\textcolor[rgb]{0.00,0.00,0.00}{#1}}
\newcommand{\VerbatimStringTok}[1]{\textcolor[rgb]{0.31,0.60,0.02}{#1}}
\newcommand{\WarningTok}[1]{\textcolor[rgb]{0.56,0.35,0.01}{\textbf{\textit{#1}}}}
\usepackage{graphicx}
\makeatletter
\def\maxwidth{\ifdim\Gin@nat@width>\linewidth\linewidth\else\Gin@nat@width\fi}
\def\maxheight{\ifdim\Gin@nat@height>\textheight\textheight\else\Gin@nat@height\fi}
\makeatother
% Scale images if necessary, so that they will not overflow the page
% margins by default, and it is still possible to overwrite the defaults
% using explicit options in \includegraphics[width, height, ...]{}
\setkeys{Gin}{width=\maxwidth,height=\maxheight,keepaspectratio}
% Set default figure placement to htbp
\makeatletter
\def\fps@figure{htbp}
\makeatother
\setlength{\emergencystretch}{3em} % prevent overfull lines
\providecommand{\tightlist}{%
  \setlength{\itemsep}{0pt}\setlength{\parskip}{0pt}}
\setcounter{secnumdepth}{-\maxdimen} % remove section numbering
\ifLuaTeX
  \usepackage{selnolig}  % disable illegal ligatures
\fi

\begin{document}
\maketitle

\hypertarget{group-member}{%
\subsection{Group Member:}\label{group-member}}

\begin{itemize}
\tightlist
\item
  Cecheng Chen(cchen549)
\item
  Zhuocheng Sun(zsun273)
\item
  Boya Zeng (bzeng7)
\item
  Yueyu Wang(wang2537)
\item
  Zihan Zhu (zzhu338)
\end{itemize}

\hypertarget{abstract}{%
\subsection{Abstract}\label{abstract}}

Today, women earn approximately 82 cents for every dollar earned by a
man(Carlton, G., 2021, The biggest barriers for women in STEM). However,
a report from Scientific American shows that in the STEM field, males
and females have approximately equal average base salaries (Ceci et al.,
2015, Scientific American). Unlike other working fields, it seems that
the gender salary gap in STEM fields is the least obvious. However,
during our research and feedback from our friends, it seems that there
still exists some salary gap in gender. Also, we found that in the STEM
field, other factors such as regions, employee' education backgrounds,
etc. also seem correlated with base salaries. Therefore, we want to
figure out whether there truly exists gender differences in the STEM
field, and if there exists, does such difference correlate with other
factors like regions, job positions, different kinds of companies, etc?
Based on these questions, our data is mainly focusing on the different
personal situations and different salaries of employee in the STEM
fields. Based on our research, we find that there still exists gender
base salary gap in STEM field, and such gender base salary gap also
depends on factors like regions (in state level), Job titles (positions)
and Companies.

\hypertarget{dataset-descriptions}{%
\subsection{Dataset descriptions}\label{dataset-descriptions}}

We totally use two data sets for our program

\hypertarget{dataset1-data-science-and-stem-salaries}{%
\subsubsection{Dataset1: Data Science and STEM
Salaries}\label{dataset1-data-science-and-stem-salaries}}

The URL for this data set:
\url{https://www.kaggle.com/jackogozaly/data-science-and-stem-salaries}

This data set was scraped off levels.fyi by Jack Ogozaly. levels.fyi is
a website that lets you compare career levels \& compensation packages
across different tech companies, and is generally considered more
accurate in terms of actual tech salaries relative to other compensation
sites like glassdoor.com. We use this dataset because this dataset has
most information we want: The base salary, gender, company, location,
races, etc of an employee. We can easily use these information to
compare median base salary differences in gender and compare such
difference based on other factors. With this data set, we can answer our
research questions more easily and efficiently.

\hypertarget{description-of-this-dataset}{%
\paragraph{Description of this
dataset:}\label{description-of-this-dataset}}

This dataset contains 62,000 salary records from top STEM companies like
Amazon, Apple, Google, SpaceX, etc. For each salary record, it contains
the company employee works for, job titles and positions, base salary,
total year salary, gender, race, and other useful information. We can
use this data set to analyze the income situation of workers in these
companies based on the characteristics of personal and companyies.

Variables :

\begin{itemize}
\tightlist
\item
  Timestamp: When the data was recorded.
\item
  Company: The company name where the employee works (Google, Facebook,
  etc)
\item
  Level: What level the employee is at.
\item
  Title: Role title.
\item
  Total yearly compensation: Total yearly compensation.
\item
  Location: Job location.
\item
  Years of experience: Years of Experience.
\item
  Years at company: Years of experience at said company.
\item
  Tag: Job type
\item
  Base salary: Base salary an employee earned in a year
\item
  Stock grant value: The equivalent value for the stocks the employee
  received.
\item
  Bonus: These bonuses could be in the form of a lump sum cash payment,
  increment cash payments, stock options, or even an added vacation, we
  only count the bonus in dollar amounts here.
\item
  Gender: gender identity of employee (male, female or other)
\item
  Other details: Other details for the employee
\item
  City id: The id for the city where the employee works
\item
  Dmaid: Designated Market Areas (DMAs) delineate the geographic
  boundaries of 210 distinctive regions to assess TV penetration of
  audience counts within the U.S. for a viewership year.
\item
  Row Number: row number of the data entry
\item
  Masters\_Degree: Whether the employee has a Master Degree (1: Yes, 0:
  No)
\item
  Bachelors\_Degree: Whether the employee has a Bachelor Degree (1: Yes,
  0: No)
\item
  Doctorate\_Degree: Whether the employee has a Doctor Degree (1: Yes,
  0: No)
\item
  Highschool: Whether the employee has a High school Degree (1: Yes, 0:
  No)
\item
  Some\_College: if the employee has education limited to some college
  education.
\item
  Race\_Asian: if the employee is asian (1: Yes, 0: No)
\item
  Race\_White: if the employee is white (1: Yes, 0: No)
\item
  Race\_Two\_Or\_More: if the employee has two or more races. (1: Yes,
  0: No)
\item
  Race\_Black: if the employee is black (1: Yes, 0: No)
\item
  Race\_Hispanic: if the employee is hispanic (1: Yes, 0: No)
\item
  Race: Racial identity of the employee
\item
  Education: the Education level of employee
\end{itemize}

\hypertarget{dataset2-cost-of-living-index-by-state-2021}{%
\subsubsection{Dataset2: Cost of Living Index by State
2021}\label{dataset2-cost-of-living-index-by-state-2021}}

The URL for this data set:
\url{https://worldpopulationreview.com/state-rankings/cost-of-living-index-by-state}

This data set is from and gathered by World Population Review website
that allows you to compare the cost of living index for different
states. We used this dataset because for some states like CA, WA, NY and
MA are very expensive states and many tech companies are based there, so
base salaries will be higher there, and that may nagetively impact our
regression model if we use state as one possible predictor. Therefore,
to best predict base salary of different states and minimize such
effects, we plan to add the cost of living index by state variable as an
addition predictor.

\hypertarget{description-of-this-dataset-1}{%
\paragraph{Description of this
dataset:}\label{description-of-this-dataset-1}}

This data set contains the cost-of-living-index for each state in
America, also contains the cost index in sub living categories like
Grocery, Housing, Utilities, etc.

Variables:

\begin{itemize}
\tightlist
\item
  Cost Index: The overall cost of living index for each state in
  America, the higher the index is, the higher overall living expense in
  that state.
\item
  Grocery: The cost of index in Grocery category for each state in
  America
\item
  Housing: The cost of index in Housing category for each state in
  America
\item
  Utilities: The cost of index in Utilities category for each state in
  America
\item
  Transportation: The cost of index in Transportation category for each
  state in America
\item
  Misc: The cost of index in misc category for each state in America
\end{itemize}

\hypertarget{statistical-questions}{%
\subsection{Statistical Questions:}\label{statistical-questions}}

Main question: Is there a gender gap in income level in different data
science and STEM subfields and different regions? Does the difference
vary depending on other factors (e.g., education, subfields, companies,
regions, etc.)?

\hypertarget{process-and-summaries}{%
\section{Process and summaries:}\label{process-and-summaries}}

Because in our data set may contain outliners (i.e Some CEOs and
employees may have extremely high base salaries), so instead of using
average base salary, our research focus on the median base salaries.

At first, we want to consider whether there truly exists gender gap in
base salary, so we drow a box plot to show the median income (in
logarithm) of males and females.

\begin{verbatim}
## Warning: Removed 823 rows containing non-finite values (stat_boxplot).
\end{verbatim}

\includegraphics{stat340-project_files/figure-latex/unnamed-chunk-6-1.pdf}

According to the above plot, It seems that the median income of females
is only slightly lower than males income. Therefore, we cannot tell
whether males and females truly have different median income in stem
fields, so we conduct hypothesis tests to verify what is shown in the
plot.

Our null hypothesis is
\(H_0 : Median\_Salary_{male} = Median\_Salary_{female}\), and the
alternative hypothesis is
\(H_1: Median\_Salary_{male} \neq Median\_Salary_{female}\)

First, we conduct a two sample wilcoxon test, a test focus on testing
the median of datasets, and the test result shows as follows:

\begin{verbatim}
## 
##  Wilcoxon rank sum test with continuity correction
## 
## data:  basesalary by gender
## W = 78023286, p-value < 2.2e-16
## alternative hypothesis: true location shift is not equal to 0
\end{verbatim}

Based on the p value of the test result, which is smaller than 2.2e-16,
which is much smaller than 0.01 and is highly significant. We have
strong statistical evidence to reject the null hypothesis in favor of
the conclusion that the median income for males and females are not the
same.

We also apply the Monte carol testing to test the null hypothesis.
Firstly, we randomly assign the income data into male group and female
groups, and compare their medians, store the median income difference as
an element in a list Replicate . Then we repeat the step mentioned above
1000 times and use all elements in Replicate to generate a 95\%
confidence interval. Finally we test whether the true median income
difference is in the confidence interval.

Base on the Monte carol testing, our 95\% confidence intercal is

\begin{verbatim}
##       2.5%      97.5% 
## -3530.8770  -239.7871
\end{verbatim}

and the true income difference is \texttt{-9000}, which is not in the
confidence interval. Therefore, we can reject the null hypothesis that
the median income of males and females are the same.

Based on the hypothesis tests, we find that in STEM fields, there truly
exists gender base salary gap in STEM fields, which answers our first
questions. So our next step is to consider whether the gender salary
difference vary depending on other factors.

Firstly, we want to figure out what factors are correlation to the base
salary of employees in STEM field, so we fit an linear regression model
to predict base salary of employees.

\begin{verbatim}
## 
## Call:
## lm(formula = basesalary ~ 1 + gender + yearsofexperience + yearsatcompany + 
##     title + costIndex + new_location, data = filted_data)
## 
## Residuals:
##     Min      1Q  Median      3Q     Max 
## -283306  -16461    -615   17252 1465575 
## 
## Coefficients: (1 not defined because of singularities)
##                                     Estimate Std. Error t value Pr(>|t|)    
## (Intercept)                        435901.21  873135.73   0.499   0.6176    
## genderMale                           4424.31     652.47   6.781 1.21e-11 ***
## yearsofexperience                    4162.13      50.32  82.718  < 2e-16 ***
## yearsatcompany                      -1138.13      85.02 -13.386  < 2e-16 ***
## titleData Scientist                 37280.52    2518.76  14.801  < 2e-16 ***
## titleHardware Engineer              21812.48    2554.35   8.539  < 2e-16 ***
## titleHuman Resources                 5268.65    4043.10   1.303   0.1925    
## titleManagement Consultant          28782.48    3056.92   9.416  < 2e-16 ***
## titleMarketing                      13366.12    3213.21   4.160 3.19e-05 ***
## titleMechanical Engineer             8730.61    3503.62   2.492   0.0127 *  
## titleProduct Designer               22734.36    2645.01   8.595  < 2e-16 ***
## titleProduct Manager                28631.04    2396.05  11.949  < 2e-16 ***
## titleRecruiter                      -3789.92    3647.61  -1.039   0.2988    
## titleSales                          -3268.09    3943.68  -0.829   0.4073    
## titleSoftware Engineer              28797.05    2248.61  12.807  < 2e-16 ***
## titleSoftware Engineering Manager   41489.91    2483.53  16.706  < 2e-16 ***
## titleSolution Architect             16158.52    2921.16   5.532 3.20e-08 ***
## titleTechnical Program Manager      22733.34    2763.38   8.227  < 2e-16 ***
## costIndex                           -4288.71    9709.63  -0.442   0.6587    
## new_locationAR                     -18756.18   30989.88  -0.605   0.5450    
## new_locationAZ                      35933.11   69292.87   0.519   0.6041    
## new_locationCA                     319106.64  599912.80   0.532   0.5948    
## new_locationCO                      85252.59  152483.26   0.559   0.5761    
## new_locationCT                     167885.72  366955.51   0.458   0.6473    
## new_locationDE                      82999.72  176834.14   0.469   0.6388    
## new_locationFL                      38583.85   77982.79   0.495   0.6208    
## new_locationGA                       8730.02   11096.96   0.787   0.4315    
## new_locationHI                     483612.12 1000429.80   0.483   0.6288    
## new_locationIA                       2497.04   12333.74   0.202   0.8396    
## new_locationID                      11029.70   26266.38   0.420   0.6745    
## new_locationIL                      37415.26   45271.11   0.826   0.4085    
## new_locationIN                       -271.66    9909.16  -0.027   0.9781    
## new_locationKS                      -7833.54   14612.39  -0.536   0.5919    
## new_locationKY                       3635.10   16273.25   0.223   0.8232    
## new_locationLA                      13608.38   40703.91   0.334   0.7381    
## new_locationMA                     203953.74  404779.00   0.504   0.6144    
## new_locationMD                     189047.58  386365.79   0.489   0.6246    
## new_locationME                     113108.71  269157.61   0.420   0.6743    
## new_locationMI                       2875.80   13418.98   0.214   0.8303    
## new_locationMN                      62956.80  113750.25   0.553   0.5799    
## new_locationMO                     -16742.00   28863.46  -0.580   0.5619    
## new_locationMS                     -14660.53   49461.04  -0.296   0.7669    
## new_locationMT                      62345.74  166080.26   0.375   0.7074    
## new_locationNC                      37189.07   49115.55   0.757   0.4490    
## new_locationND                      39405.92   92212.10   0.427   0.6691    
## new_locationNE                       -906.04   15908.63  -0.057   0.9546    
## new_locationNH                      89807.89  192465.99   0.467   0.6408    
## new_locationNJ                     177535.14  341689.89   0.520   0.6034    
## new_locationNM                       5952.19   30118.92   0.198   0.8433    
## new_locationNV                      71623.61  180877.99   0.396   0.6921    
## new_locationNY                     260694.22  477587.31   0.546   0.5852    
## new_locationOH                       8444.72   12541.24   0.673   0.5007    
## new_locationOK                     -20325.11   32247.78  -0.630   0.5285    
## new_locationOR                     209671.94  430021.35   0.488   0.6258    
## new_locationPA                      66318.50  114700.44   0.578   0.5631    
## new_locationRI                     129185.73  286598.88   0.451   0.6522    
## new_locationSC                      29584.61   59587.12   0.496   0.6195    
## new_locationTN                      -5616.65   15583.69  -0.360   0.7185    
## new_locationTX                      22333.39   17471.10   1.278   0.2012    
## new_locationUT                      47205.95   82820.00   0.570   0.5687    
## new_locationVA                      68716.09  105007.48   0.654   0.5129    
## new_locationVT                     106608.63  239499.01   0.445   0.6562    
## new_locationWA                     125500.01  201932.18   0.621   0.5343    
## new_locationWI                      46701.70   72258.26   0.646   0.5181    
## new_locationWV                            NA         NA      NA       NA    
## new_locationWY                      21415.20   45889.03   0.467   0.6407    
## ---
## Signif. codes:  0 '***' 0.001 '**' 0.01 '*' 0.05 '.' 0.1 ' ' 1
## 
## Residual standard error: 44730 on 34925 degrees of freedom
## Multiple R-squared:  0.3008, Adjusted R-squared:  0.2996 
## F-statistic: 234.8 on 64 and 34925 DF,  p-value: < 2.2e-16
\end{verbatim}

According to the regression model,males have a high potential ability to
earn more base salary, and some job titles, years of experiences also
affect the base salary of employees. However, it seems like state
location do not effects the employees, so do location effects the gender
base salary gap?

Based on the regression model, we want to figure out whether the gender
base salary gap very in factors liek job title, location (in state
level) and companies.

First we focused on the location factors, we want to figure out whether
regions (at state level) influence the income difference of males and
females. We compute the income difference for each state and make the
following frequency plot.

\includegraphics{stat340-project_files/figure-latex/unnamed-chunk-11-1.pdf}
The overall histogram shows that the values range from around -20000 to
near 40000 with a normal distribution, which means for many states,
median income for males is 10 thousand dollars greater than females, but
there are also plenty of states that have more serious income
differences and some states have smaller income differences. Therefore,
it seem that at the state level, there exists base salary differences of
males and females.

To verity there exists base salary differences of males and females in
state level, we conduct a chi-squared test to verity it. If there is no
obvious gender base salary gap in state level, the stand deviation of
the income difference should be very small (close to 0)

Our null hypothesis is \$H\_0: \$ no obvious gender base salary gap in
state level and \(H_1:\) There exists gender base salary gap in state
level

\begin{verbatim}
## 
##  Chi-Squared Test on Variance
## 
## data:  state_rela$sal_diff
## Chi-Squared = 8.5671e+13, df = 42, p-value < 2.2e-16
## alternative hypothesis: true variance is not equal to 1e-04
## 95 percent confidence interval:
##  138678637 329521430
## sample estimates:
##  variance 
## 203978959
\end{verbatim}

According to the chi-squared test, because the p-value is smaller than
\texttt{2.2e-16}, which means the we can reject other null hypothesis
that there is no obvious gender base salary gap in state level.

Then we want to figure out which states have gender base salary gap, and
which states do not have. So for each state, we conduct a two sample
wilcoxon test to test whether the median base salary of male and female
in that state is equal. Then we show the test result as follows:

\begin{verbatim}
##    Row State      P_Value significant
## 1    1    CA 8.330696e-38        TRUE
## 2    2    WA 3.004607e-33        TRUE
## 3    3    MA 1.390441e-05        TRUE
## 4    4    NY 2.182609e-19        TRUE
## 5    5    SC 6.225668e-01       FALSE
## 6    6    OR 4.728207e-04        TRUE
## 7    7    VA 1.647016e-01       FALSE
## 8    8    CO 1.453442e-01       FALSE
## 9    9    NE 1.594100e-01       FALSE
## 10  10    PA 8.402782e-01       FALSE
## 11  11    IN 6.037835e-01       FALSE
## 12  12    WI 5.124009e-02       FALSE
## 13  13    TX 1.493283e-02        TRUE
## 14  14    MN 3.346728e-02        TRUE
## 15  15    IL 1.088802e-01       FALSE
## 16  16    NJ 1.555425e-01       FALSE
## 17  17    AZ 3.752096e-01       FALSE
## 18  18    NC 4.118402e-03        TRUE
## 19  19    CT 6.301570e-01       FALSE
## 20  20    NM 5.714286e-01       FALSE
## 21  21    GA 8.977133e-01       FALSE
## 22  22    FL 9.473036e-03        TRUE
## 23  23    UT 1.920187e-01       FALSE
## 24  24    AR 4.763092e-01       FALSE
## 25  25    VT 1.000000e+00       FALSE
## 26  26    IA 1.000000e+00       FALSE
## 27  27    KS 5.016907e-01       FALSE
## 28  28    MI 8.499536e-01       FALSE
## 29  29    OH 2.505470e-01       FALSE
## 30  30    NH 3.483951e-01       FALSE
## 31  31    MD 4.147890e-01       FALSE
## 32  32    TN 3.107200e-01       FALSE
## 33  33    MO 1.366325e-01       FALSE
## 34  34    DE 8.184499e-01       FALSE
## 35  35    AL 1.000000e+00       FALSE
## 36  36    ID 3.820266e-01       FALSE
## 37  37    NV 7.365355e-01       FALSE
## 38  38    KY 2.538700e-01       FALSE
## 39  39    RI 4.547610e-01       FALSE
## 40  40  <NA>           NA          NA
## 41  41  <NA>           NA          NA
## 42  42    OK 5.039434e-02       FALSE
## 43  43    ME 1.000000e+00       FALSE
## 44  44    MT 2.666667e-01       FALSE
## 45  45    MS 1.000000e+00       FALSE
## 46  46  <NA>           NA          NA
## 47  47  <NA>           NA          NA
## 48  48  <NA>           NA          NA
\end{verbatim}

\begin{verbatim}
## # A tibble: 2 x 2
##   significant     n
##   <lgl>       <int>
## 1 FALSE          34
## 2 TRUE            9
\end{verbatim}

According to the results of our wilcoxon test, we can find that based on
our data, there are 9 states reject our null hypothesis that there are
no base salary difference. And data from 34 states cannot reject the
null hypothesis. Ststes that reject the null hypothesis are as follows:

\begin{verbatim}
##   Row State      P_Value significant
## 1   1    CA 8.330696e-38        TRUE
## 2   2    WA 3.004607e-33        TRUE
## 3   3    MA 1.390441e-05        TRUE
## 4   4    NY 2.182609e-19        TRUE
## 5   6    OR 4.728207e-04        TRUE
## 6  13    TX 1.493283e-02        TRUE
## 7  14    MN 3.346728e-02        TRUE
## 8  18    NC 4.118402e-03        TRUE
## 9  22    FL 9.473036e-03        TRUE
\end{verbatim}

Therefore, at least based on our data and test results, there is obvious
evidence that CA, WA, MA, NY, OR, TX, MN, NC, and FL states has gender
base salary gaps, which means the the base salary difference in gender
are vary related to location.

\begin{Shaded}
\begin{Highlighting}[]
\NormalTok{filted\_data}\SpecialCharTok{$}\NormalTok{company }\OtherTok{=} \FunctionTok{toupper}\NormalTok{(filted\_data}\SpecialCharTok{$}\NormalTok{company)}
\NormalTok{companies }\OtherTok{\textless{}{-}}\NormalTok{ filted\_data }\SpecialCharTok{\%\textgreater{}\%} \FunctionTok{group\_by}\NormalTok{(company)}\SpecialCharTok{\%\textgreater{}\%}
  \FunctionTok{summarise}\NormalTok{(}\AttributeTok{n=} \FunctionTok{n}\NormalTok{())}

\NormalTok{com\_rela }\OtherTok{\textless{}{-}}\NormalTok{ filted\_data }\SpecialCharTok{\%\textgreater{}\%}
  \FunctionTok{group\_by}\NormalTok{(company,gender)}\SpecialCharTok{\%\textgreater{}\%}
  \FunctionTok{summarise}\NormalTok{(}\AttributeTok{basesalary=}\FunctionTok{median}\NormalTok{(basesalary)) }\SpecialCharTok{\%\textgreater{}\%}
  \FunctionTok{pivot\_wider}\NormalTok{(}\AttributeTok{names\_from =}\NormalTok{ gender, }\AttributeTok{values\_from =}\NormalTok{ basesalary) }\SpecialCharTok{\%\textgreater{}\%}
  \FunctionTok{drop\_na}\NormalTok{() }\SpecialCharTok{\%\textgreater{}\%}
  \FunctionTok{mutate}\NormalTok{(}\AttributeTok{sal\_diff=}\NormalTok{ Male }\SpecialCharTok{{-}}\NormalTok{Female)}
\end{Highlighting}
\end{Shaded}

\begin{verbatim}
## `summarise()` has grouped output by 'company'. You can override using the `.groups` argument.
\end{verbatim}

\begin{Shaded}
\begin{Highlighting}[]
\FunctionTok{hist}\NormalTok{(}\FunctionTok{log}\NormalTok{(com\_rela}\SpecialCharTok{$}\NormalTok{sal\_diff))}
\end{Highlighting}
\end{Shaded}

\begin{verbatim}
## Warning in log(com_rela$sal_diff): 产生了NaNs
\end{verbatim}

\includegraphics{stat340-project_files/figure-latex/unnamed-chunk-15-1.pdf}

\begin{Shaded}
\begin{Highlighting}[]
\FunctionTok{mean}\NormalTok{(com\_rela}\SpecialCharTok{$}\NormalTok{sal\_diff)}
\end{Highlighting}
\end{Shaded}

\begin{verbatim}
## [1] 11093.65
\end{verbatim}

\end{document}
